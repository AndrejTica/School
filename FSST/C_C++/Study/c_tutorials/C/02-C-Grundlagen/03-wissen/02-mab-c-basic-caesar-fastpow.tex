% Created 2019-02-24 So 17:07
% Intended LaTeX compiler: pdflatex
\documentclass[11pt]{article}
\usepackage[utf8]{inputenc}
\usepackage[T1]{fontenc}
\usepackage{graphicx}
\usepackage{grffile}
\usepackage{longtable}
\usepackage{wrapfig}
\usepackage{rotating}
\usepackage[normalem]{ulem}
\usepackage{amsmath}
\usepackage{textcomp}
\usepackage{amssymb}
\usepackage{capt-of}
\usepackage{hyperref}
\usepackage{listings}
\author{Anton Hofmann}
\date{\today}
\title{}
\hypersetup{
 pdfauthor={Anton Hofmann},
 pdftitle={},
 pdfkeywords={},
 pdfsubject={},
 pdfcreator={Emacs 26.1 (Org mode 9.1.9)},
 pdflang={English}}
\begin{document}

\tableofcontents

\section{probe-mab: C-basic}
\label{sec:org163fe42}

\section{AUFGABE: caesar.c}
\label{sec:org38f6b17}
\begin{itemize}
\item gegeben:
\end{itemize}
\lstset{language=C,label= ,caption= ,captionpos=b,numbers=none}
\begin{lstlisting}
//mycat.c  Eingabe nach Ausgabe
#include <stdio.h>
int main(){
	int ch;
	while ( (ch= fgetc(stdin)) != EOF){
		fputc(ch, stdout);
	}

	return 0;
}
\end{lstlisting}
\begin{itemize}
\item Erstelle das Programm caesar.c
\begin{itemize}
\item Es wird das gesamte Alphabet um eine bestimmte Anzahl von Buchstaben \textbf{verschoben} und
\item dadurch jeder Buchstabe des Klartextes einzeln verändert.
\item Der "Key" besteht also aus einer Zahl, um die der ASCII-Wert des Plain-Zeichens erhöht wird.
\end{itemize}
\item Vorgaben:
\begin{itemize}
\item Es sollen nur Buchstaben verschlüsselt werden.
\item Verwende im Programm die Variable: int key=5;
\item Verwende zum Testen die Dateiumlenkung: ./caesar.exe < caesar.c
\end{itemize}
\item Zusatz:
\begin{itemize}
\item Können Sie die Datei: werwolf-caesarkodiert.txt knacken?
\end{itemize}

\item Beispiel: Cäsar verschlüsseln
\begin{itemize}
\item Plaintext: "HALLO"
\item Key: 2
\item Ciphertext: JCNNQ
\end{itemize}

\item Beachte:
\begin{itemize}
\item Wenn man zB. zum Zeichen 'Z' kommt muss man beim Zeichen 'A' weiter zählen. (analog z -> a)
\item Wenn das jeweilige CIPHERzeichen > 'Z' (analog für 'z')
\item dann CiPHERzeichen = CIPHERzeichen  - 26
\end{itemize}

\item Beispiel:
\end{itemize}
\lstset{language=C,label= ,caption= ,captionpos=b,numbers=none}
\begin{lstlisting}
ch=ch+key;
if(ch>'Z')
    ch= ch-26;

bzw. für Kleinbuchstaben

ch=ch+key;
if(ch>'z')
    ch= ch-26;
\end{lstlisting}

\begin{itemize}
\item Anmerkung: Entschlüsselt wird mit key=26-key;

\item Beispiel:
\begin{itemize}
\item Hello, world!		(key=4)
\item Lipps, asvph! 		(entschluesselt wird mit 22) (vgl: 26  - 4)
\end{itemize}
\end{itemize}


\section{AUFGABE: fastpow.c}
\label{sec:org687ecee}
\begin{itemize}
\item x hoch y kann man wie folgt sehr schnell berechnen:
\item wenn y ungerade:
\begin{itemize}
\item ergebnis= ergebnis*x;
\item y= y-1;
\end{itemize}
\item wenn y gerade:
\begin{itemize}
\item x= x*x;
\item y= y/2;
\end{itemize}

\item Beispiel: x=3 und y=16
\begin{itemize}
\item ergebnis= 3 hoch 16
\item ergebnis= (3 hoch 2) hoch 8
\end{itemize}

\item Beispiel: x=3 und y=17
\begin{itemize}
\item ergebnis= 3 hoch 17
\item ergebnis= 3 hoch 16 * 3
\item ergebnis*3=3 hoch 16
\end{itemize}

\item Erstelle das Programm fastpow.c
\begin{itemize}
\item lies x und y ein
\item gib das Ergebnis von x hoch y aus
\end{itemize}
\end{itemize}
\end{document}
